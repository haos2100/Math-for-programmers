\myheading{LCM}

Many people use \ac{LCM} in school. Sum up $\frac{1}{4}$ and $\frac{1}{6}$.
To find an answer mentally, you ought to find Lowest Common Denominator, which can be 4*6=24.
Now you can sum up $\frac{6}{24} + \frac{4}{24} = \frac{10}{24}$.

But the lowest denominator is also a LCM.
LCM of 4 and 6 is 12: $\frac{3}{12} + \frac{2}{12} = \frac{5}{12}$.

\leveldown{}

\myheading{File copying routine}

\epigraph{Buffer: A storage device used to compensate for a difference in data rate of data flow or time of occurrence of events, when transmitting data from one device to another.}
{Clarence T. Jones, S. Percy Jones -- Patrick-Turner's Industrial Automation Dictionary}

In GNU coreutils, we can find that LCM is used to find optimal buffer size, if buffer sizes in input and ouput files are differ.
For example, input file has buffer of 4096 bytes, and output is 6144.
Well, these sizes are somewhat suspicious. I made up this example.
Nevertheless, LCM of 4096 and 6144 is 12288. This is a buffer size you can allocate, so that you will minimize number of read/write operations during copying.

\url{https://github.com/coreutils/coreutils/blob/4cb3f4faa435820dc99c36b30ce93c7d01501f65/src/copy.c#L1246}.
\url{https://github.com/coreutils/coreutils/blob/master/gl/lib/buffer-lcm.c}.

\levelup{}

