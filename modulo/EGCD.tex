\myheading{Getting magic number using extended Euclidean algorithm}

Extended Euclidean algorithm can find x/y for given a/b in the diophantine equation:
$ax+by=gcd(a,b)$.

x/y are also known as ``Bézout coefficients''.

However, since a/b are coprime to each other, $gcd(a,b)=1$, and the algorithm will find
x/y for the $ax+by=1$ equation.

Let's plug $a=3$ and $b=2^{16}$ (like if we finding magic constant for 16-bit CPU):

\lstinputlisting[style=customc]{modulo/EGCD.c}

\begin{lstlisting}
GCD: 1
Bézout coefficients: -21845 1
quotients by the GCD (s,t): 65536 -3
corrected coefficients: 43691(0xaaab) 2(0x2)
\end{lstlisting}

That means, x=-21845, y=1.
Is it correct?
$-21845*3 + 65536*1=1$

65536-21845=0xaaab, and this is the magic number for division by 3 on 16-bit CPU.

$GCD(any\_odd\_number, 2^n)=1$ for any values.
Hence, we can find a magic number for any even number.
But, this is not true for even numbers.
We can't find magic coefficient for even number like 10.
But you can find it for 5, and then add additional division by 2 using shift right.

If you try to compile $\frac{x}{10}$ code, GCC can do it like:

\begin{lstlisting}
push   %ebp
mov    %esp,%ebp
mov    0x8(%ebp),%eax
mov    $0xcccccccd,%edx
mul    %edx
mov    %edx,%eax
shr    $0x3,%eax
pop    %ebp
ret    
\end{lstlisting}

This is in fact the magic number for division by 5.
And there is 3 instead of 2 in the SHR instruction, so the result is divided by 2.

Extended Euclidean algorithm is probably an efficient way of finding magic number, but needless to say,
this equation can be solved using other ways.
In \SSBE you can find a method of finding "magic number" using SMT-solver.

