\myheading{Binary logarithm}

Sometimes denoted as lb(), binary logarithms are prominent in computer science, 
because numbers are usually stored and processed in computer in binary form.

\leveldown{}

\myheading{Denoting a number of bits for some value}

How many bits we need to allocate to store googol number ($10^{100}$)?

\begin{lstlisting}[caption=Wolfram Mathematica]
In[]:= Log2[10^100] // N
Out[]= 332.193
\end{lstlisting}

Binary logarithm of some number is the number of how many bits needs to be allocated.

If you have a variable which always has $2^x$ form, it's a good idea to store a binary logarithmic representation ($\log_2 (x)$) instead of it.
There are at least two reasons:
1) the programmer shows to everyone that the number has always $2^x$ form;
2) it's error-prone, it's not possible to accidentally store a number in some other form to this variable, so this is some kind of protection;
3) logarithmic representation is more compact.
There is, however, performance issue: the number must be converted back, but this is just one shifting operation (\texttt{1<<log\_n}).

Here is an example from NetBSD NTP client ( \texttt{netbsd-5.1.2/usr/src/dist/ntp/include/ntp.h} ):

\begin{lstlisting}[caption=C code]
/*
 * Poll interval parameters
 */
...
#define NTP_MINPOLL     4       /* log2 min poll interval (16 s) */
#define NTP_MINDPOLL    6       /* log2 default min poll (64 s) */
#define NTP_MAXDPOLL    10      /* log2 default max poll (~17 m) */
#define NTP_MAXPOLL     17      /* log2 max poll interval (~36 h) */
\end{lstlisting}

Couple examples from zlib (\texttt{deflate.h}):

\begin{lstlisting}[caption=C code]
    uInt  w_size;        /* LZ77 window size (32K by default) */
    uInt  w_bits;        /* log2(w_size)  (8..16) */
    uInt  w_mask;        /* w_size - 1 */
\end{lstlisting}

Another piece from zlib ( \texttt{contrib/blast/blast.c} ):

\begin{lstlisting}[caption=C code]
    int dict;           /* log2(dictionary size) - 6 */
\end{lstlisting}

If you need to generate bitmasks in range 1, 2, 4, 8...0x80000000, it is good idea to assign self-documenting name to iterator variable:

\begin{lstlisting}[caption=C code]
for (log2_n=1; log2_n<32; log2_n++)
    1<<log2_n;
\end{lstlisting}

Now about compactness, here is the fragment I found in OpenBSD, related to SGI IP22 architecture
\footnote{\url{http://www.linux-mips.org/wiki/IP22}} \\
( \texttt{OS/OpenBSD/sys/arch/sgi/sgi/ip22\_machdep.c} ):

\begin{lstlisting}[caption=C code]
                /*
                 * Secondary cache information is encoded as WWLLSSSS, where
                 * WW is the number of ways
                 *   (should be 01)
                 * LL is Log2(line size)
                 *   (should be 04 or 05 for IP20/IP22/IP24, 07 for IP26)
                 * SS is Log2(cache size in 4KB units)
                 *   (should be between 0007 and 0009)
                 */
\end{lstlisting}

Here is another example of using binary logarithm in Mozilla JavaScript engine (JIT compiler)
\footnote{\url{http://fossies.org/linux/seamonkey/mozilla/js/src/jit/mips/CodeGenerator-mips.cpp}}.
If some number is multiplied by $2^x$, the whole operation can be replaced by bit shift left.
The following code ( \texttt{js/src/jit/mips/CodeGenerator-mips.cpp} ), 
when translating multiplication operation into MIPS machine code, first, get assured if the number 
is really has $2^x$ form, then it takes binary logarithm of it and generates MIPS SLL instruction, which states for ``Shift Left Logical''.

\begin{lstlisting}[caption=Mozilla JavaScript JIT compiler (translating multiplication operation into MIPS bit shift instruction)]
bool
 CodeGeneratorMIPS::visitMulI(LMulI *ins)
 {
           default:
             uint32_t shift = FloorLog2(constant);
 
             if (!mul->canOverflow() && (constant > 0)) {
                 // If it cannot overflow, we can do lots of optimizations.
                 uint32_t rest = constant - (1 << shift);
 
                 // See if the constant has one bit set, meaning it can be
                 // encoded as a bitshift.
                 if ((1 << shift) == constant) {
                     masm.ma_sll(dest, src, Imm32(shift));
                     return true;
                 }

...

\end{lstlisting}

Thus, for example, $x=y \cdot 1024$ (which is the same as $x=y \cdot 2^{10}$) translates into $x=y<<10$.

\myheading{Calculating binary logarithm}

If all you need is integer result of binary logarithm ($abs(\log_2(x))$ or $\lfloor \log_2(x) \rfloor$), 
calculating is just counting all binary digits in the number minus 1.
In practice, this is the task of calculating leading zeroes.

Here is example from Mozilla libraries ( \texttt{mfbt/MathAlgorithms.h}
\footnote{\url{http://fossies.org/linux/seamonkey/mozilla/mfbt/MathAlgorithms.h}} ):

\begin{lstlisting}[caption=Mozilla libraries]
class FloorLog2<T, 4>
{
public:
   static uint_fast8_t compute(const T aValue)
   {
     return 31u - CountLeadingZeroes32(aValue | 1);
   }
 };

inline uint_fast8_t
 CountLeadingZeroes32(uint32_t aValue)
 {
   return __builtin_clz(aValue);
 }
\end{lstlisting}

Latest x86 CPUs has LZCNT (Leading Zeroes CouNT) instruction for that
\footnote{GNU \texttt{\_\_builtin\_clz()} function on x86 architecture can be thunk for LZCNT}, 
but there is also BSR (Bit Scan Reverse) instruction appeared in 80386, which can be used for the same purpose.
More information about this instruction on various architectures: \url{https://en.wikipedia.org/wiki/Find_first_set}.

There are also quite esoteric methods to count leading zeroes without this specialized instruction: \url{http://yurichev.com/blog/de_bruijn/}.

\myheading{O(log n) time complexity}

Time complexity\footnote{\url{https://en.wikipedia.org/wiki/Time_complexity}} is a measure of speed of a specific algorithm 
in relation to the size of input data.

O(1) -- time is always constant, to matter what size of input data. Simplest example is object getter -- it just returns some value.

O(n) -- time is linear, growing according to the size of input data. Simplest example is search for some value in the input array.
The larger array, the slowest search.

O(log n) -- time is logarithmic to the input data.
Let's see how this can be.

Let's recall child's number guessting game\footnote{\url{http://rosettacode.org/wiki/Guess_the_number}}.
One player think about some number, the other should guess it, offering various versions.
First player answers, is guessed number is larger or less.
Typical dialogue can be as the follows:

\begin{lstlisting}
-- I think of a number in 1..100 range.
-- Is it 50?
-- My number is larger.
-- 75?
-- It is lesser.
-- 63?
-- Larger.
-- 69?
-- Larger.
-- 72?
-- Lesser.
-- 71?
-- Correct.
\end{lstlisting}

Best possible strategy is to divide the range in halves.
The range is shorten at each step by half.
At the very end, the range has lenght of 1, and this is correct answer.
Maximal number of steps using the strategy described here are $\log_2(initial\_range)$.
In our example, initial range is 100, so the maximum number of steps is 6.64... or just 7.
If the initial range is 200, maximum number of steps are $\log_2(200)=7.6..$ or just 8. The number of steps increasing by 1 when the range is doubled.
Indeed, doubled range indicates that the guesser needs just one more step at the start, not more.
If the initial range is 1000, numbers of steps are $\log_2(1000)=9.96...$ or just 10.

This is exactly O(log n) time complexity.\\
\\
Now let's consider couple of practical real-world algorithms.
One interesting thing is that if the input array is sorted, and its size is known, and we need to find some value in it, the algorithm works exactly
in the same way as child's number guessing game!
The algorithm starts in the middle of array and compare the value there with the value sought-after.
Depending on the result (larger or lesser), the \textit{cursor} is moved left or right and operating range is decreasing by half.
This is called binary search\footnote{\url{https://en.wikipedia.org/wiki/Binary_search_algorithm}}, and there is the \texttt{bsearch()} function in 
standard C/C++ library\footnote{\url{http://en.cppreference.com/w/cpp/algorithm/bsearch}}.\\
\\
Here is how binary search is used in git: \url{https://www.kernel.org/pub/software/scm/git/docs/git-bisect.html}.\\
\\
Another prominent example in CS is binary trees. They are heavily used internally in almost any programming language, when you use set, map, dictionary, etc.

Here is a simple example with the following numbers (or keys) inserted into binary tree:
0, 1, 2, 3, 5, 6, 9, 10, 11, 12, 20, 99, 100, 101, 107, 1001, 1010.

\tikzset
{
  treenode/.style = {align=center, inner sep=0pt, text centered},
  key/.style = {treenode, circle, white, draw=black, fill=black, minimum size=2.0em, text width=2.0em},
  empty/.style = {treenode, rectangle, draw=black, minimum width=0.5em, minimum height=0.5em}
}

\begin{center}
\begin{tikzpicture}[->,>=stealth',
		level 1/.style={sibling distance=6cm},
		level 2/.style={sibling distance=2.7cm},
		level 3/.style={sibling distance=1.7cm},
		level 4/.style={sibling distance=0.7cm}] 
	\node [key] {10}
	child { node [key] {1} 
		child{ node [key] {0} 
		}
		child{ node [key] {5}
			child{ node [key] {3}
				child{ node[key] {2}}
				child{ node[empty] {}}
			}
			child{ node [key] {6}
				child{ node[empty] {}}
				child{ node[key] {9}}
			}
		}                            
	}
	child { node [key] {100}
		child { node [key] {20}
			child{ node [key] {12}
				child{ node[key] {11}}
				child{ node[empty] {}}
			}
			child{ node [key] {99}
			}
		}
		child{ node [key] {107}
			child{ node [key] {101}}
			child{ node [key] {1001}
				child{ node[empty] {}}
				child{ node[key] {1010}}
			}
		}
	}
	; 
\end{tikzpicture}
\end{center}


And here is how binary tree search works: put \textit{cursor} at the root. Now compare the value under it with the value sought-after.
If the value we are seeking for is lesser than the current, take a move into left node. If it's bigger, move to the right node.
Hence, each left descendant node has value lesser than in 
ascendant node.
Each right node has value which is bigger.
The tree must be rebalanced after each modification (I gave examples of it in my book about reverse engineering (\url{http://beginners.re/}, 51.4.4)).
Nevertheless, lookup function is very simple, and maximal number of steps is $\log_n(number\_of\_nodes)$.
We've got 17 elements in the tree at the picture, $\log_2(17)=4.08...$, indeed, there are 5 tiers in the tree.

\levelup{}

