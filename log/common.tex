\myheading{Common (base 10) logarithms}

Also known as ``decimal logarithms''.
Denoted as \texttt{lg} on handheld calculators.

10 is a number inherently linked with human's culture, since almost all humans has 10 digits.
Decimal system is a result of it.
Nevertheless, 10 has no special meaning in mathematics and science in general.
So are common logarithms.

One notable use is a decibel logarithmic scale, which is based of common logarithm.

Common logarithms are sometimes used to calculate space for decimal number in the string or on the screen.
How many characters you should allocate for 64-bit number? 20, because $log_{10}(2^{64}) = 19.2...$.

Functions like \texttt{itoa()}\footnote{\url{http://www.cplusplus.com/reference/cstdlib/itoa/}}
(which converts input number to a string) can calculate output buffer size precisely, 
calculating common logarithm of the input number.

