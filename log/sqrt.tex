\myheading{Square root}

Likewise, square root can be computed in the following way:

\begin{equation}
\sqrt[2]{x} = 2^{\frac{\log_2{x}}{2}}
\end{equation}

This leads to interesting consequence: if you have a value stored in logarithmical form and you need to take
square root of it and leave it in logarithmical form, all you need is just to divide it by 2.

And since floating point numbers encoded in IEEE 754 has exponent encoded in logarithmical form,
you need just to shift it right by 1 bit to get square root:
\url{https://en.wikipedia.org/wiki/Methods_of_computing_square_roots#Approximations_that_depend_on_the_floating_point_representation}.

Likewise, cube root and nth root can be calculated using logarithm of corresponding base:

\begin{equation}
\sqrt[b]{x} = b^{\frac{\log_b{x}}{b}}
\end{equation}

