\section{Introduction}

\subsection{Children's approach}

When children argue about how big their favorite numbers are, they speaking about how many zeroes it has:
``$x$ has $n$ zeroes!''
``No, my $y$ is bigger, it has $m>n$ zeroes!''

This is exactly notion of common (base 10) logarithm.

Googol ($10^{100}$) has 100 zeroes, so $log_{10} (googol) = 100$.

Let's take some big number, like 12th Mersenne prime:

\begin{lstlisting}[caption=Wolfram Mathematica]
In[]:= 2^127 - 1
Out[]= 170141183460469231731687303715884105727
\end{lstlisting}

Wow, it's so big. How can we measure it in childish terms? How many digits it has? We can count using common (base 10) logarithm:

\begin{lstlisting}[caption=Wolfram Mathematica]
In[]:= Log[10, 2^127 - 1] // N
Out[]= 38.2308
\end{lstlisting}

So it has 39 digits.

Another question, how may decimal digits 1024-bit RSA key has?

\begin{lstlisting}[caption=Wolfram Mathematica]
In[]:= 2^1024
Out[]= 17976931348623159077293051907890247336179769789423065727343008\
1157732675805500963132708477322407536021120113879871393357658789768814\
4166224928474306394741243777678934248654852763022196012460941194530829\
5208500576883815068234246288147391311054082723716335051068458629823994\
7245938479716304835356329624224137216

In[]:= Log10[2^1024] // N
Out[]= 308.255
\end{lstlisting}

309 decimal digits.

\subsection{Scientists' and engineers' approach}

Interestingly enough, scientists' and engineers' approach is not very different from children's.
They are not interesting in noting each digit of some big number, they are usually interesting in three properties of some number:
1) sign; 2) first $n$ digits (significand or mantissa); 3) exponent (how many digits the number has).

The common way to represent a real number in handheld calculators and FPUs is:

\begin{equation}
(sign) significand \times 10^{exponent}
\end{equation}

For example:

\begin{equation}
-1.987126381 \times 10^{41}
\end{equation}

It was common for scientific handheld calculators to use the first ~10 digits of significand and ignore everything behind.
Storing the whole number down to the last digit is 1) very expensive; 2) hardly useful.

The number in IEEE 754 format (most popular way of representing real numbers in computers) has these three parts, however, 
it has different base (2 instead of 10).

